\documentclass{article}

\usepackage{graphicx}
\usepackage{amsmath}

\title{APG4005F Assignment 3 - Free Network Adjustment}
\date{07/05/2015}
\author{Jason David Russell - RSSJAS005}

\begin{document}

\maketitle
\pagenumbering{gobble}

\newpage
\tableofcontents

\pagenumbering{arabic}

\newpage
\section{Introduction}
The aim of this assignment is to conduct an Epoch deformation analysis using
fictitious data with a Free Network least squares adjustment.


\section{Background}
\subsection{Classification of Deformation Analysis}
There are three main classifications of deformation analysis monitoring methods,
Permanent, Semi-permanent and Epoch. There are advantages and disadvantages of all
three methods. The main advantages of the permanent and Semi-permanent methods
are that they are continuous and offer a very high precision. These two methods
make use of a multitude of sensors, such as capacitive, strain, inductance and
electro-optical sensors. These sensors are able to produce data in real time which
is useful for situations in which immediate data is required in order to, for
example, raise an alarm. Some of the disadvantages of these two methods of
deformation analysis is that the senors are expensive, and require regular
calibration. Epoch monitoring involves geodetic and/or photogramerric techniques
to capture data, this is beneficial in that relative and/or absolute positions of
many points can be obtained, as apposed to just relative positions in the case
of the Permanent and Semi-permanent methods mentioned above. Another advantage
of Epoch monitoring is that it is much more cost effective.

\subsection{Network Classifications}
Typically, when constructing a network for Epoch deformation analysis,
a free or minimum constrained network is used, preferably free. In a free
network adjustment, no parameter is held fixed, and as a result, precision
estimates for all points are provided in the variance-covariance matrices.
The effect of holding no parameters fixed is that the shape of the network
is defined only by the observations. One of the main advantages of not holding
any parameters fixed is that the shape of the network is not affected by erros in
the coordinates of the points defining the datum (because the network is not tied
to the datum and is allowed to `float'). Free networks are especially useful
in cases where precise surveys are connected to existing point coordinates
of lower precisions. A caveat of the free network adjustment is that because the
datum is not defined (no points are fixed) a singular normal equation matrix
will occur (a rank defect occurs in the normal equation matrix).
As singular matrices have a determinant of zero (because one of the
eigen values is zero) the normal equation matrix cannot be inversed, and so
a solution vector `x' cannot be obtained. In order to negate the singular normal
equation matrix, special mathematical treatment based on the determination of a
generalized inverse is applied. To remove the singularity in the normal equation
matrix, a set of pseudo-observation equations are added to to the normal
equation matrix in such a way that these equations remove the singularity and
do not affect the result vector `x'.

\subsection{Concepts of deformation analysis using geodetic method}

\subsubsection{Points selection}
Points representing the feature to be monitored are selected.
The placement of deformation control points must adhere to two criteria:
The network configuration must follow the principles of conventional geodetic
network design, and the selected points must be representative of the feature
in question and points should be suitably placed so as to effectively detect and
model any deformation.

\subsubsection{Network}
A network pre-analyisis based on the least squares adjustment theory should be
undertaken. Then, a conventional network of appropriate accuracy should be
executed in at least two epochs. In terms of coordinate system to use, Local
systems are adequate and preferable, as larger scale coordinate systems are
usually of a poorer precision.

\subsubsection{Deformation Models}
One of a variety of available deformation analysis techniques based on statistical
testing is used to detect if point deformations have occurred. Often, a second
method is employed to confirm the first analysis.

\subsubsection{Inspection}
The quantities and directions of deformations are determined.

\section{Problem Statement}
The aim of this assignment was to conduct an epoch based, deformation analysis
using a free network adjustment with an appropriate deformation analysis 
tequinique.

\section{Method}
Data for a free network adjustment was obtained after performing a network
pre-analysis. Four control points were used to observe to various points 
of intereset. Two epochs were observed, in the second epoch, points of intereset
were artificailly displaced in order to repersent a deformation of those points.

\paragraph{}
A free network adjustment was then carried out on the xy and the z values
independantly.

\paragraph{}
The X-method of deformation analysis was to be employed in order to model the 
deformation. The X-method relies on the comparason of point coordinates for the 
various epochs. The basis of the comparason is the assumption that any deformations
in a point field will result in shifts of points which are then reflected in a 
change of the xy coordinates of the displaced points.
\paragraph{}
This method like many other analysis methods attempts to answer three questions:
1. Have points shifted, 2. Which points have shifted, 3. What form does the deformation 
have.
\paragraph{}
Another popular deformation modelling technique is the l-method. The l-methor or the
invarient functions method compares those quantities of the various epoch networks
which are invarient to a change of the reference system. Distances and angles will
remain unchanged with a change of the datum.

\section{Conclusion}
In designing a deformation analysis survey, it is important to decide upon the most
effective type of deformation analysis to undertake, whether it be permanent or epoch 
based. It is also very important to select points which are representitive of the area
of a feature which is suspect to deform. Assistance from specialists of other fields such as civil
engineers should always be seeked so as to ensure an effective deformation survey.

\end{document}
