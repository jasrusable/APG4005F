\documentclass{article}

\usepackage{graphicx}
\usepackage{amsmath}

\title{APG4005F Assignment 3 - Free Network Adjustment}
\date{07/05/2015}
\author{Jason David Russell - RSSJAS005}

\begin{document}

\maketitle
\pagenumbering{gobble}

\newpage
\tableofcontents

\pagenumbering{arabic}

\newpage
\section{Introduction}
The aim of this assignment is to conduct an Epoch deformation analysis using
ficticious data with a Free Network least squares adjustment.


\section{Background}
\subsection{Classification of Deformation Analysis}
There are three main classifications of deformation analysis monitoring methods,
Permanent, Semi-permanet and Epoch. There are advantages and disadvatages of all
three methods. The main advatages of the Permenatn and Semi-permanent methods
are that they are continous and offer a very high precsion. These two methods
make use of a multitude of sensors, such as capacitive, strain, inductance and
elctro-optical sensors. These sensors are able to produce data in realtime which
is useful for situations in which immediate data is required in order to, for
example, raise an alarm. Some of the disadvatages of these two methods of
deformation analyis is that the senors are expensive, and require regular
calibration. Epoch monitoring involves geodetic and/or photogramerric techniqies
to capture data, this is benefitial in that relative and/or absolut postions of
many points can be obtained, as apposed to just relative postions in the case
of the Permanent and Semi-permanent methods mentioned above. Another advantage
of Epoch monitering is that it is much more cost effective.

\subsection{Network Classifications}
Typically, when constructing a network for Epoch deformation anaylis,
a free or minimum constrained network is used, perferably free. In a free
network adjustment, no paramter is held fixed, and as a result, precision
estimates for all points are provided in the variance-covariance matrices.
The effect of holding no paramters fixed is that the shape of the network
is defined only by the observations. One of the main advantages of not holding
any paramters fixed is that the shape of the network is not affected by erros in
the coordintes of the points defining the datum (because the network is not tied
to the datum and is allowed to 'float'). Free netowkrs are espicially useful
in cases where precise surveys are connected to existing point coordiantes
of lower preceions. A caveat of the free network adjustment is that because the
datum is not defined (no points are fixed) a singular normal equation matrix
will occur (a rank defect occurs in the normal equation matrix).
As singular matrices have a determinant of zero (because one of the
eigen values is zero) the normal equation matrix cannot be inversed, and so
a solution vector 'x' cannot be obtained. In order to negate the singular normal
equation matrix, special mathematical treatment based on the determinateion of a
generalized inverse is applied.

\section{Problem Statement}

\section{Method}

\section{Results}

\section{Discussion}

\section{Conclusion}

\end{document}
